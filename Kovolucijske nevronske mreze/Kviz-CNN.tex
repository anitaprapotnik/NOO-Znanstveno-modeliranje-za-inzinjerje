\documentclass[12pt]{exam}
\usepackage[slovene]{babel}
\usepackage{graphicx}

\usepackage{ucs}
\usepackage[utf8x]{inputenc}
\usepackage{pgfplots}

%\usepackage{arev}
%\usepackage[T1]{fontenc}


%\renewcommand*\rmdefault{iwona}

%\renewcommand*{\familydefault}{\rmdefault}
%\renewcommand*{\familydefault}{\sfdefault}
%\renewcommand*{\familydefault}{\ttdefault}

%\textwidth 16 true cm
%\textheight 27 true cm
%\oddsidemargin 30 true mm
%\evensidemargin 20 true mm
%\voffset= -18 true mm
%\hoffset= -25 true mm


\printanswers

\begin{document}
\pagestyle{empty}

\section*{Kviz: CNN}
\begin{questions}

\question Zakaj so konvolucijske nevronske mreže (CNN) na splošno bolj primerne za obdelavo slik kot gosto povezane mreže?
\begin{choices}
\choice Ker vsak vhodni piksel povežejo z nevronom v prvi plasti, kar omogoča bolj celovito analizo.
\choice Ker za svoje delovanje ne potrebujejo aktivacijskih funkcij.
\correctchoice Ker ohranjajo prostorsko razporeditev pikslov in upoštevajo njihovo medsebojno lego.
\choice Ker imajo manj plasti in so zato računsko hitrejše.
\end{choices}

\question Kakšna je razlika v delovanju med \emph{max pooling} in \emph{average pooling} plastjo?
\begin{choices}
\choice Max pooling se uporablja samo za barvne slike, average pooling pa za črno-bele.
\choice Max pooling poveča velikost slike, average pooling pa jo zmanjša.
\correctchoice Max pooling izpostavi najmočnejši signal, medtem ko average pooling podatke bolj izravna.
\choice Max pooling izračuna povprečne vrednosti, average pooling pa izbere največjo.
\end{choices}


\question Kateri korak je opisan v postopku prilagajanja pred-učeneega VGG modela za lastne potrebe?
\begin{choices}
\correctchoice Odstranitev zadnje klasifikacijske plasti in njena zamenjava s plastjo, prilagojeno našemu številu kategorij.
\choice Odstranitev vseh konvolucijskih plasti in ohranitev samo gosto povezanih.
\choice Zamenjava prve konvolucijske plasti z novo, prilagojeno velikosti naših slik.
\choice Povečanje števila filtrov v vseh konvolucijskih plasteh za dvakrat.
\end{choices}


\question Kaj pomeni izraz »deljenje uteži« (weight sharing) pri konvolucijski plasti?
\begin{choices}
\choice Da več različnih filtrov v isti plasti uporablja enake uteži za iskanje različnih značilnosti.
\choice Da se uteži filtra razdelijo med vse nevrone v naslednji plasti.
\correctchoice Da en in isti filter (z istimi utežmi) drsi po celotni sliki in se uporablja na različnih lokacijah.
\choice Da se uteži iz ene konvolucijske plasti delijo z utežmi v naslednji pooling plasti.
\end{choices}


\question Kateri podatkovni nabor je tako obsežen, da zaradi licenčnih omejitev ni več prosto dostopen prek Kerasa?
\begin{choices}
\choice MNIST
\correctchoice ImageNet
\choice KITTI Dataset
\choice CIFAR-100
\end{choices}


\question V primeru iz besedila »Marko kosi travo, Mateja pa kosi juho«, kaj omogoča plast večglave pozornosti v transformerju?
\begin{choices}
\choice Popravi slovnično napako, če ta obstaja.
\choice Prevede stavek v drug jezik.
\correctchoice Ugotovi, da se glagol »kosi« nanaša na različne stvari in pravilno interpretira pomen.
\choice Prepozna, da sta Marko in Mateja imeni oseb.
\end{choices}


\question Kakšna je tipična zgradba osnovne konvolucijske nevronske mreže?
\begin{choices}
\choice Samo zaporedje konvolucijskih plasti, ki vodijo neposredno do izhoda.
\choice Goste plasti → sploščevanje → izmenično konvolucijska in pooling plast.
\correctchoice Izmenično konvolucijska plast in plast za združevanje → sploščevanje → goste plasti.
\choice Sploščevalna plast → goste plasti → konvolucijska plast.
\end{choices}


\question Kateri dve lastnosti sta ključni za delovanje konvolucijske plasti?
\begin{choices}
\choice Max pooling in average pooling.
\choice Sploščevanje in gosta povezanost.
\choice Dodajanje robov (padding) in korak (stride).
\correctchoice Lokalna povezanost in deljenje uteži.
\end{choices}

\question Kateri je glavni namen plasti za združevanje (pooling layer) v CNN?
\begin{choices}
\choice Povečanje dimenzij podatkov za boljšo prepoznavo.
\correctchoice Zmanjšanje količine podatkov ob ohranjanju najpomembnejših informacij.
\choice Uporaba filtrov za iskanje robov in kontrastov.
\choice Pretvorba večdimenzionalnih podatkov v enodimenzionalni vektor.
\end{choices}


\question Če želimo, da je izhodna slika po konvoluciji enake velikosti kot vhodna, katero strategijo dodajanja robov uporabimo?
\begin{choices}
\choice Circular padding
\correctchoice Same padding
\choice Valid padding
\choice Full padding
\end{choices}


\question Kakšna je vloga sploščevalne plasti (Flatten layer) v arhitekturi CNN?
\begin{choices}
\correctchoice Pretvori podatke iz metrične oblike v enodimenzionalni vektor.
\choice Združi majhne skupine podatkov v eno vrednost.
\choice Izvede končno klasifikacijo slike.
\choice Doda ničle okoli robov slike.
\end{choices}


\question Katera trditev najbolje opisuje arhitekturo VGG modelov?
\begin{choices}
\choice Uporablja velike 7×7 filtre.
\correctchoice Zaporedja 3×3 konvolucijskih plasti, ki jim sledijo 2×2 max pooling plasti.
\choice Število filtrov se z globino zmanjšuje.
\choice Nimajo gosto povezanih plasti.
\end{choices}


\question Kateri podatkovni nabor je zbirka črno-belih, ročno pisanih številk velikosti 28×28 pikslov?
\begin{choices}
\correctchoice MNIST
\choice ImageNet
\choice CIFAR-10
\choice Udacity Self-Driving Car Dataset
\end{choices}


\question Kaj je glavna značilnost transformer modelov, ki jih loči od CNN in gosto povezanih mrež?
\begin{choices}
\correctchoice Plast pozornosti (attention layer), ki tehta pomembnost različnih delov vhoda.
\choice Izključna uporaba gosto povezanih plasti.
\choice Sposobnost obdelave slik visoke ločljivosti.
\choice Uporaba konvolucijskih filtrov za zaporedja.
\end{choices}


\question Kako deluje metoda dodajanja robov »mirror padding«?
\begin{choices}
\choice Okoli robov doda piksle z vrednostjo nič.
\choice Podatke obravnava ciklično.
\choice Odreže robne piksle.
\correctchoice Zrcali vrednosti pikslov ob robu.
\end{choices}


\end{questions}

\newpage
\section*{Naloga: CNN}


\begin{enumerate}
\item   Uporabite podatkovno zbirko CIFAR-10 iz zgradite CNN, ki bo znal razpoznavati kategorije iz te knjižnice.
\item   Poiščite nekaj slik iz interneta ali pa jih sami posnamite, da preverite delovanje modela na vaših slikah.
\item {\em Dodatno:} Uporabite podatkovne zbirke, ki jih najdete na internetu ter oblikujte in strenirajte CNN model. Preverite kako se obnese pri razpoznavanju vaših slik ali slik z interneta.
\end{enumerate}



\end{document}
