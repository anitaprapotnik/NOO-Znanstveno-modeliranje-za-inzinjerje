\documentclass[12pt]{exam}
\usepackage[slovene]{babel}
\usepackage{graphicx}

\usepackage{ucs}
\usepackage[utf8x]{inputenc}
\usepackage{pgfplots}

%\usepackage{arev}
%\usepackage[T1]{fontenc}


%\renewcommand*\rmdefault{iwona}

%\renewcommand*{\familydefault}{\rmdefault}
%\renewcommand*{\familydefault}{\sfdefault}
%\renewcommand*{\familydefault}{\ttdefault}

%\textwidth 16 true cm
%\textheight 27 true cm
%\oddsidemargin 30 true mm
%\evensidemargin 20 true mm
%\voffset= -18 true mm
%\hoffset= -25 true mm


\printanswers

\begin{document}
\pagestyle{empty}

\section*{Kviz: Naključna števila}

\begin{questions}

\question Kaj je temeljna značilnost enakomerne porazdelitve, po kateri osnovni algoritem v programskih jezikih generira števila med 0 in 1?
\begin{choices}
\correctchoice Verjetnost izbire števila v kateremkoli podintervalu je odvisna le od širine tega podintervala.
\choice Vsako naslednje generirano število je večje od prejšnjega.
\choice Algoritem generira le števila, ki imajo končno število decimalnih mest.
\choice Verjetnost, da je število blizu 0,5, je največja.
\end{choices}

\question Kaj se zgodi, če ukaz \verb|numpy.random.rand()| pokličete brez argumentov?
\begin{choices}
\correctchoice Vrne eno samo (skalarno) število med 0 in 1.
\choice Vrne prazno polje (array) dolžine 0.
\choice Program javi napako, ker je argument obvezen.
\choice Uporabi privzeto število 100 in vrne 100 številk.
\end{choices}

\question Kateri dve ključni lastnosti mora imeti dober algoritem za generiranje psevdonaključnih števil?
\begin{choices}
\choice Generirana števila sledijo Gaussovi porazdelitvi in so vedno pozitivna.
\choice Algoritem je zelo hiter in porabi malo pomnilnika.
\correctchoice Števila so porazdeljena po enakomerni porazdelitvi in so nekorelirana.
\choice Zaporedje je kratko in hitro ponovljivo.
\end{choices}

\question Kaj je najpogostejši vir za določitev začetnega semena, če ga uporabnik ne določi?
\begin{choices}
\choice Prvo število, ki ga uporabnik vnese preko tipkovnice.
\correctchoice Trenutni sistemski čas, npr.\ odčitek notranje ure v milisekundah.
\choice Vrednost števila pi.
\choice Vnaprej določeno število, enako za vse računalnike.
\end{choices}

\question Katere tri parametre sprejme funkcija \verb|numpy.random.normal()|?
\begin{choices}
\choice Število poskusov, verjetnost uspeha in število števil.
\correctchoice Povprečno vrednost, standardni odklon in število števil.
\choice Stopnjo, obliko in število števil.
\choice Spodnjo mejo, zgornjo mejo in število števil.
\end{choices}

\question Katera metoda uporablja veliko število naključnih števil za reševanje determinističnih problemov?
\begin{choices}
\choice Optimizacija z naključnim iskanjem.
\choice Analiza meritev in napak.
\correctchoice Monte Carlo metode.
\choice Inverzna transformacijska metoda.
\end{choices}

\question Kdaj je priporočljivo odstraniti \verb|numpy.random.seed()| iz končne različice programa?
\begin{choices}
\correctchoice Ko je program preizkušen in želimo, da ob vsakem zagonu ustvari novo, neponovljivo zaporedje naključnih števil.
\choice Ko program deluje prepočasi, saj določanje semena upočasni izvajanje.
\choice Samo takrat, ko uporabljamo porazdelitve, ki niso enakomerne.
\choice Nikoli, saj vedno zagotavlja boljšo naključnost.
\end{choices}

\question Katera trditev najbolje povzema uporabo naključnih števil pri simulacijah?
\begin{choices}
\correctchoice Uporabljajo se za modeliranje pojavov, kjer ima naključnost ključno vlogo.
\choice Služijo iskanju optimalnih parametrov tehničnih sistemov.
\choice Namen je ocena merilnih napak.
\choice Uporabljajo se za reševanje analitično težkih, a determinističnih problemov.
\end{choices}

\question Zakaj se v testiranju uporabi \verb|numpy.random.seed(N)|?
\begin{choices}
\choice Poveča hitrost generiranja naključnih števil.
\choice Za generiranje bolj kakovostnih naključnih števil.
\correctchoice Da program ob vsakem zagonu ustvari identično zaporedje naključnih števil.
\choice Da omeji obseg generiranih števil med 0 in N.
\end{choices}

\question Kaj pomeni, da so psevdonaključna števila določena z determinističnim procesom?
\begin{choices}
\choice Računalnik uporablja kvantnomehanske procese.
\choice Števila so le približki realnih števil.
\correctchoice Zaporedje števil je vnaprej določeno z začetnim semenom.
\choice Zaporedja se ponavljajo že po nekaj tisoč številih.
\end{choices}

\question Kateri numpy ukaz generira polje 10 števil, ki sledijo binomski porazdelitvi?
\begin{choices}
\choice numpy.random.normal(n,p,10)
\correctchoice numpy.random.binomial(n,p,10)
\choice numpy.random.poisson(10)
\choice numpy.random.rand(10)
\end{choices}

\question Za katero nalogo je Monte Carlo najbolj primeren?
\begin{choices}
\choice Reševanje sistema linearnih enačb.
\choice Razvrščanje podatkov.
\choice Iskanje besede v besedilu.
\correctchoice Izračun števila $\pi$ z naključnimi točkami v kvadratu.
\end{choices}

\question Kaj je namen dodajanja naključnega šuma v simulaciji meritev?
\begin{choices}
\choice Pospešiti zbiranje podatkov.
\correctchoice Oceniti negotovost meritve zaradi napak posameznih meritev.
\choice Odpraviti vse meritvene napake.
\choice Preveriti delovanje inštrumentov.
\end{choices}

\question Na kateri funkciji temelji inverzna transformacijska metoda?
\begin{choices}
\choice Standardnem odklonu.
\choice Gostoti verjetnosti (PDF).
\choice Karakteristični funkciji.
\correctchoice Kumulativni porazdelitveni funkciji (CDF).
\end{choices}

\question Zakaj je optimizacija z majhnimi naključnimi odmiki boljša od slepega iskanja?
\begin{choices}
\choice Ker vedno najde globalni optimum.
\choice Ker je računsko manj zahtevna.
\choice Ker ne potrebuje začetne rešitve.
\correctchoice Ker raziskuje okolico dobre rešitve namesto celotnega prostora.
\end{choices}

\question Kateri pojav bi modelirali z naključnim generiranjem?
\begin{choices}
\choice Delovanje logičnih vrat v procesorju.
\choice Izračun vsote prvih 100 števil.
\choice Gibanje planeta okoli Sonca.
\correctchoice Jedrski razpad, kjer je čas razpada nepredvidljiv.
\end{choices}

\end{questions}

\newpage

\section*{Naloga: Naključna števila}

Prosim, oddajte Jupyter Notebook datoteko z opravljenimi naslednjimi nalogami:
\begin{enumerate}
\item     Napišite program, ki s pomočjo Monte Carlo metode izračuna površino med x-osjo in parabolo z enačbo $𝑦=1−𝑥^2$ 
med -1 in 1.
\item Za zgornjo parabolo zgenerirajte signal s šumom in poiščite parabolo, ki se ji najbolj prilega.
\item Podobno kot v prejšnji nalogi, vendar namesto, da šum generirate po normalni porazdelitvi, uporabite Poissonovo in/ali eksponentno porazdelitev.
\item {\em Dodatno:} Napišite program za vajo deljenja števil do 100 (»obratna poštevanka«).

\end{enumerate}




\end{document}
