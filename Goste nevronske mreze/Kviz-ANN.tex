\documentclass[12pt]{exam}
\usepackage[slovene]{babel}
\usepackage{graphicx}

\usepackage{ucs}
\usepackage[utf8x]{inputenc}
\usepackage{pgfplots}

%\usepackage{arev}
%\usepackage[T1]{fontenc}


%\renewcommand*\rmdefault{iwona}

%\renewcommand*{\familydefault}{\rmdefault}
%\renewcommand*{\familydefault}{\sfdefault}
%\renewcommand*{\familydefault}{\ttdefault}

%\textwidth 16 true cm
%\textheight 27 true cm
%\oddsidemargin 30 true mm
%\evensidemargin 20 true mm
%\voffset= -18 true mm
%\hoffset= -25 true mm


\printanswers

\begin{document}
\pagestyle{empty}

\section*{Kviz: ANN}
\begin{questions}

\question Kateri od naslednjih opisov najbolje povzema hierarhični odnos med umetno inteligenco (UI), nevronskimi mrežami in velikimi jezikovnimi modeli (LLM), kot je ChatGPT, glede na priloženo besedilo?
\begin{choices}
\choice Nevronske mreže so se razvile iz velikih jezikovnih modelov in predstavljajo osnovo za umetno inteligenco.
\choice Umetna inteligenca je zgolj teoretični koncept, medtem ko so nevronske mreže in LLM praktične implementacije.
\choice UI, nevronske mreže in LLM so tri medsebojno ločena področja raziskav.
\CorrectChoice Veliki jezikovni modeli so podkategorija nevronskih mrež, te pa so podkategorija znotraj širšega področja umetne inteligence.
\end{choices}

\question Model za klasifikacijo je za pravilen razred (drugi od treh) napovedal verjetnost 0.2. Vrednost križne entropije je bila \(H=-\log(0.2)=1.609\). Kaj bi se zgodilo z vrednostjo križne entropije, če bi model za isti primer napovedal verjetnost 0.8 za pravilen razred?
\begin{choices}
\CorrectChoice Vrednost križne entropije bi se zmanjšala.
\choice Vrednost križne entropije bi se povečala.
\choice Vrednost križne entropije bi ostala enaka.
\choice Vrednost križne entropije bi postala negativna.
\end{choices}

\question Katera trditev o arhitekturi nevronske mreže je pravilna glede na besedilo?
\begin{choices}
\choice V osnovni (gosti) arhitekturi je vsako vozlišče povezano le z enim vozliščem sosednje plasti.
\choice Število skritih plasti in število vozlišč v njih je točno določeno s številom vhodnih plasti.
\choice Uteži so predpisani nevronom, odmiki pa povezavam med njimi.
\CorrectChoice Nevronska mreža mora imeti vsaj eno vhodno, eno skrito in eno izhodno plast.
\end{choices}

\question Kateri optimizacijski algoritem velja za enega najpogosteje uporabljenih, saj dobro deluje v večini primerov brez veliko prilagajanja in izračunava povprečje gradientov ter njihovih kvadratov?
\begin{choices}
\choice RMSProp
\choice Naključni gradientni spust (SGD)
\choice AdaGrad
\CorrectChoice Adam
\end{choices}

\question Kaj je glavna razlika med stroškovno funkcijo (npr. MSE) in metriko (npr. natančnost)?
\begin{choices}
\choice Ni nobene razlike, gre za sopomenki, ki se uporabljata za merjenje uspešnosti modela.
\choice Stroškovna funkcija ima vedno vrednost med 0 in 1, medtem ko metrika lahko zavzame poljubno vrednost.
\choice Metrika se uporablja za regresijske probleme, stroškovna funkcija pa za klasifikacije.
\CorrectChoice Stroškovna funkcija se uporablja za prilagajanje uteži med učenjem, medtem ko metrika služi le za končno oceno uspešnosti modela.
\end{choices}

\question V besedilu je omenjena analogija za lažje razumevanje aktivacijskih funkcij. S čim lahko sestavimo poljubno krivuljo, če nam to ne uspe s kombinacijo celih premic?
\begin{choices}
\CorrectChoice S kombinacijo poltrakov.
\choice S kombinacijo točk.
\choice S kombinacijo sinusnih funkcij.
\choice S kombinacijo parabol.
\end{choices}

\question Kaj je "batch size" v kontekstu učenja nevronske mreže?
\begin{choices}
\choice Število epoh potrebnih za konvergenco modela.
\choice Skupno število podatkov, ki so na voljo za učenje.
\choice Število nevronov v največji skriti plasti.
\CorrectChoice Velikost vzorca (število podatkov), ki gre skozi model, preden se uteži prilagodijo.
\end{choices}

\question Kdaj je umetna inteligenca postala samostojno raziskovalno področje znotraj znanstvene sfere?
\begin{choices}
\choice Po letu 2010, z vzponom velikih jezikovnih modelov.
\choice V osemdesetih letih, z razvojem nevronskih mrež.
\CorrectChoice Leta 1956.
\choice V 19. stoletju z idejami Charlesa Babbagea.
\end{choices}

\question Kaj je namen normalizacije zveznih vhodnih podatkov pred učenjem nevronske mreže?
\begin{choices}
\CorrectChoice Preslikava podatkov v manjši interval (npr. od 0 do 1), da imajo vsi vhodi primerljivo vrednost in se preprečijo numerične težave.
\choice Pretvorba podatkov v kategorije z uporabo one-hot kodiranja.
\choice Odstranitev izstopajočih vrednosti (outlierjev) iz nabora podatkov.
\choice Povečanje števila učnih vzorcev, za boljše učenje.
\end{choices}

\question Kateri problem bi bil primeren za reševanje z nevronsko mrežo, kot je opisano v primerih uporabe v besedilu?
\begin{choices}
\choice Shranjevanje velikih količin podatkov o strankah v struktuirano bazo podatkov.
\choice Izračun vsote prvih 1000 praštevil.
\choice Razvrščanje seznama imen po abecedi.
\CorrectChoice Napovedovanje verjetnosti prometne nesreče na križišču na podlagi gostote prometa, vidljivosti in lastnosti ceste.
\end{choices}

\question Če za klasifikacijski problem z več razredi uporabimo "one-hot" kodiranje za prave vrednosti, katero stroškovno funkcijo moramo izbrati?
\begin{choices}
\CorrectChoice Križno entropijo (cross-entropy)
\choice Redko križno entropijo (sparse cross-entropy)
\choice Adam
\choice Povprečno kvadratno napako (MSE)
\end{choices}

\question Zakaj je uporaba aktivacijskih funkcij, kot je ReLU, ključnega pomena v arhitekturi nevronskih mrež?
\begin{choices}
\CorrectChoice Ker v mrežo vnesejo nelinearnosti, kar omogoča učenje kompleksnih vzorcev, ki jih z linearnimi kombinacijami ne bi mogli zajeti.
\choice Za določitev začetnih vrednosti uteži in odmikov v mreži.
\choice Za zmanjšanje števila potrebnih nevronov v skritih plasteh in s tem poenostavitev arhitekture.
\choice Za normalizacijo vhodnih podatkov, da imajo vsi primerljivo vrednost.
\end{choices}

\question Če je natančnost modela na validacijskih podatkih precej nižja kot na učnih podatkih, je to znak za:
\begin{choices}
\choice Premalo epoh učenja.
\CorrectChoice Prekomerno prilagajanje (overfitting).
\choice Napačno izbiro stroškovne funkcije.
\choice Uspešno posploševanje modela.
\end{choices}

\question Če nevronska mreža napoveduje ceno nepremičnine (zvezni podatek), katera stroškovna funkcija je najpogosteje uporabljena?
\begin{choices}
\choice Križna entropija
\choice Redka križna entropija
\CorrectChoice Povprečna kvadratna napaka
\choice Funkcija ReLU
\end{choices}

\question Pojav, ko se nevronska mreža "na pamet nauči" učne podatke in zato slabo posplošuje na nove, nevidene podatke, imenujemo:
\begin{choices}
\CorrectChoice Prekomerno prilagajanje (overfitting)
\choice Validacija
\choice Gradientni spust
\choice Normalizacija
\end{choices}

\question Kaj se zgodi, če je hitrost učenja (learning rate) v procesu optimizacije nastavljena premajhno?
\begin{choices}
\CorrectChoice Učenje bo zelo počasno in bo morda obtičalo v lokalnem minimumu.
\choice Uteži in odmiki se sploh ne bodo posodabljali.
\choice Model se bo takoj prekomerno prilagodil (overfitting).
\choice Model bo postal nestabilen in napaka bo močno nihala.
\end{choices}

\question V vhodni plasti nevronske mreže mora biti točno toliko vozlišč, kolikor je:
\begin{choices}
\choice Epoh učenja.
\choice Skritih plasti v mreži.
\CorrectChoice Vhodnih (neodvisnih) podatkov ali značilnosti.
\choice Izhodnih spremenljivk, ki jih želimo napovedati.
\end{choices}

\question Kakšna je vrednost izraza \( a_{12}=I_1 w_{1,1}+I_2 w_{1,2}+I_3 w_{1,3}+b_{1} \), ki predstavlja izračun v nevronu prve skrite plasti?
\begin{choices}
\choice Napaka napovedi, izračunana s povprečno kvadratno napako.
\choice Produkt vseh vhodnih vrednosti in vseh uteži.
\choice Rezultat aktivacijske funkcije ReLU.
\CorrectChoice Vsota produktov vhodnih vrednosti in uteži, prišteta odmiku tega nevrona.
\end{choices}

\question Kaj je glavna pomanjkljivost uporabe celoštevilskega kodiranja (npr. rdeča → 1, črna → 2, rumena → 3) za kategorialne podatke, ki nimajo naravnega vrstnega reda?
\begin{choices}
\choice Postopek je računsko prezahteven za velike nabore podatkov.
\choice Zahteva preveč pomnilnika v primerjavi z one-hot kodiranjem.
\CorrectChoice Modelu umetno nakazuje zaporedje ali velikostni odnos, ki v resnici ne obstaja.
\choice Ni ga mogoče uporabiti za probleme z več kot dvema kategorijama.
\end{choices}

\question Kakšen je namen validacijskega nabora podatkov v procesu učenja nevronske mreže?
\begin{choices}
\choice Da se na njem neposredno prilagajajo uteži in odmiki med vsako epoho.
\choice Da se z njimi določi optimalno število skritih plasti pred začetkom učenja.
\CorrectChoice Da se oceni, kako se bo model obnesel na novih, še nevidenih podatkih.
\choice Da se vanj shranijo najboljše uteži po vsaki epohi.
\end{choices}

\end{questions}

\section*{Naloga: ANN}


\begin{enumerate}
\item Zgenerirajte tabelo s 10.000 vrsticami tako da bodo vrednosti v stolpcih A, B in C naključno generirane, vrednosti v stolpcu D pa pridobite z enačbo $2A^2+3BC$.
\item Ustvarite nevronsko mrežo s tremi skritimi plastmi, ki v vsaki skriti plasti ima 32 vozlišč in jo natrenirajte na zgoraj generiranih podatkih.
\item {\em Dodatno:} preiskusite različne optimizatorje, aktivacijske funkcije, stroškovne funkcije in matrike in zapišite opažanja.
\item {\em Dodatno:} Stolpcu D dodajte majhno Gaussovo napako in poglejte kako se natančnost modela spreminja s spreminjanjem standardne deviacije te napake.
\end{enumerate}
    
  
   \end{document}

